\newcommand{\R}{\mathbb{R}}
\newcommand{\C}{\mathbb{C}}
\newcommand{\Z}{\mathbb{Z}}
\newcommand{\N}{\mathbb{N}}
\newcommand{\Q}{\mathbb{Q}}
\newcommand{\Cdot}{\boldsymbol{\cdot}}
\newcommand{\E}{\operatorname{\mathbb{E}}}
\newcommand{\sub}{\subseteq}


\theoremstyle{definition}
\newtheorem{fact}{Fact}
\newtheorem{thm}{Theorem}
\newtheorem{defn}{Definition}
\newtheorem{rmk}{Remark}
\newtheorem{lemma}{Lemma}
\newtheorem{cor}{Corollary}

% \newenvironment{thm}[1]{%
%   \renewcommand\thethm{#1}%
%   \thm
% }{\endthm}

\section{Appetizer: Using probability to cover a geometric set}

A point $x \in \R^n$ is a \textbf{convex combination} of points $x_1,..., x_m \in \R^n$ if 
\begin{align*}
x = \sum_{i=1}^m \lambda_i x_i \qquad\text{with each } \lambda_i\geq0 \qquad\text{ and }\qquad \sum_{i=1}^m \lambda_i = 1.
\end{align*}
The \textbf{convex hull} of $T \sub \R^n$, $\text{conv}(T)$, is the set of all convex combinations of $T$.

\begin{thm}{0.0.1}[Catheodory's Theorem]\label{0.0.1}
Let $x \in \text{conv}(T)$. There exists $k \leq n+1$ points $x_1,...,x_k \in T$ such that $x$ is a convex combination of $x_1,...,x_k$.
\end{thm} 

The result says we can obtain any point in the convex hull of $T$ using at most a dimension-dependent number of points. Let the {\bf diameter} of a set $T$ be defined as $\text{diam}(T) = \sup \{\|x-y\|_2 : x,y \in T\}$.

% thm 0.0.2 - approximate catheodory's ----------

\begin{thm}{0.0.2}[Approximate Catheodory's Theorem]\label{0.0.2}
Let $\text{diam}(T) = 1$. Let $x \in \text{conv}(T)$. For any $k$, there exists $k$ points $x_1,...,x_k \in T$ such that
\begin{align*}
    \l\| x - \frac 1 k \sum_{j=1}^k x_j\r\|_2 \leq \frac 1 {\sqrt k}
\end{align*}
\end{thm}

\begin{proof}
Suppose $|T| = m$. WLOG we can assume $T$ is bounded by 1 in $\|\cdot\|_2$. We write $x = \sum_{i=1}^m\lambda_i x_i$, and interpret $\lambda_i$ as probabilities. We define the random variable
\begin{align*}
    X = x_i \text{ with probability } \lambda_i
\end{align*}
for $i=1,...,m$. We can verify that $\E X = \sum_{i=1}^m \lambda_i x_i = x$. Taking $X_1,...,X_k \overset{\text{iid}}{\sim} X$. It remains to analyse the quantity $\E \|x - \frac 1 k \sum_{j=1}^k X_j\|_2^2$.
\begin{align*}
    \E \l\| x - \frac 1 k \sum_{j=1}^k X_j\r\|_2^2 &\leq \frac{1}{k^2}\E \l\|\sum_{j=1}^k X_j -x\r\|_2^2 \\
    &= \frac{1}{k^2} \sum_{j=1}^k \E \l\| X_j - x \r\|_2^2  &\text{(by Exercise \ref{0.0.3} (a))}\\
    &= \frac 1  k \E \|X - x\|^2_2
\end{align*}
Applying the result of Exercise \ref{0.0.3} (b), we obtain
\begin{align*}
    \E \|X - x\|^2_2 = \E\|X\|_2^2 - \|\E X\|_2^2 \leq \E \|X\|_2^2 \leq 1
\end{align*}
Plugging this in above, we obtain the desired bound in expectation, hence there must exist a realization of the $X_j$, $x_1,...,x_k$, such that the bound holds.
\end{proof}

\begin{ex}{0.0.3}\label{0.0.3} Check the following identities for random vectors.
\begin{enumerate}[label=(\alph*)]

\item Let $X_1,...,X_k$ be independent, mean zero random vectors in $\R^n$. Show that
\begin{align*}
    \E \l\|\sum_{j=1}^k X_j\r\|_2^2 = \E \sum_{j=1}^k \|X_j\|_2^2
\end{align*}

\begin{proof}[Solution]
\begin{align*}
    \E \l\|\sum_{j=1}^k X_j\r\|_2^2 = \sum_{i=1}^n \E \l(\sum_{j=1}^m X_j^{(i)}\r)^2 &= \sum_{i=1}^n \Var\l(\sum_{j=1}^m X_j^{(i)}\r) &\text{(by mean zero)} \\
    &= \sum_{i=1}^n \sum_{j=1}^m \Var\l( X_j^{(i)}\r) &\text{(by independence)} \\
    &= \sum_{i=1}^n \sum_{j=1}^m\E \l(X_j^{(i)}\r)^2  &\text{(by mean zero)} \\
    &= \E\sum_{j=1}^m \|X_j\|_2^2 
\end{align*}
% \vspace*{-2\baselineskip}
\end{proof}

Among other things, this result implies that the expected squared distance of a random walk (starting from the origin) is equal to sum of the expected squared distances of each step.

\item Let $X$ be a random vector in $\R^n$. Show that
\begin{align*}
    \E \|X - \E X\|_2^2 = \E\|X\|_2^2 - \|\E X \|^2_2
\end{align*}

\begin{proof}[Solution]
\begin{align*}
    \E \|X - \E X \|_2^2 = \E \sum_{i=1}^n \l(X^{(i)}-(\E X)^{(i)}\r)^2 = \sum_{i=1}^n \Var(X^{(i)}) &= \sum_{i=1}^n\E \l(X^{(i)}\r)^2 - \l(\E X^{(i)}\r)^2 \\
    &= \E \|X\|_2^2- \|\E X\|_2^2
\end{align*}
% \vspace*{-2\baselineskip}
\end{proof}

\end{enumerate}
\end{ex}

% corollary 0.0.4 - eps-covering polytopes ----------

\begin{cor}{0.0.4}[Covering polytopes by balls]\label{0.0.4}
Let $P \sub \R^n$ be a polytope with $\text{diam}(P) = 1$. Let $m$ be the number of vertices of $P$. Let $\eps >0$.
We can cover $P$ with $m^k$ balls of radius $\eps$ for $k \geq \lceil 1/\eps^2 \rceil$.
\end{cor}

\begin{proof}
Take $T$ to be the vertex set of $P$. $|T| = m$. Note that for any $x\in P$, $x \in \text{conv}(T)$.
By Theorem \ref{0.0.2}, taking $k \geq \lceil 1/\eps^2\rceil$, we can find $x_1,...,x_k \in T$ such that
\begin{align*}
    \l\|x - \frac1k\sum_{j=1}^k x_j\r\| \leq \frac1 {\sqrt k} \leq \eps 
\end{align*}
The number of ball centres obtained from selecting a set of $k$ points out of $m$ with repetition is bounded by $m^k$ (possibly repeating orders).
Hence we have an $\eps$-cover sufficient to cover $P$.
\end{proof}

\begin{ex}{0.0.5}[Bionomial coefficient inequality]\label{0.0.5}
Show that for $1 \leq r \leq n$
\begin{align*}
    \l(\frac n r \r)^r \leq \binom{n}{r} \leq \sum_{k=0}^r \binom{n}{k} \leq \l(\frac{en}r\r)^r
\end{align*}
\end{ex}

\begin{proof}[Solution] 
For the first inequality, consider
\begin{align*}
    \frac{\l(\frac n r \r)^r}{\binom{n}{r}} = {\overbrace{\frac{\frac n r}{\frac n r}\cdot \frac{\frac n r}{\frac {n-1} {r-1}} \cdot .... \cdot \frac{\frac n r}{\frac {n-r+1} 1}}^r}
    \leq 1 \cdot 1 \cdot .... \cdot 1 = 1
\end{align*}
The second inequality follows immediately. To justify the last inequality, write
\begin{align*}
    \l(\frac {en} {r}\r)^r = e^r \cdot \l(\frac n r\r)^r &= \sum_{k=0}^\infty \frac {r^k} {k!} \cdot \l(\frac n r\r)^r &\text{(Maclaurin series for $e^x$)} \\
    &\geq \sum_{k=0}^r \frac {r^k} {k!} \cdot \l(\frac n r\r)^r \\
    &= \sum_{k=0}^r \frac {n^k \cdot n^{r-k}} {k! \cdot r^{r-k}} \\
    &\geq \sum_{k=0}^r \frac {n^k } {k!} &\text{(by $n \geq r$)}\\
    &\geq \sum_{k=0}^r \binom{n}{r}
\end{align*}
\end{proof}

\begin{ex}{0.0.6}[Improved covering]\label{0.0.6}
Show that in the setting of Corollary \ref{0.0.4}, for $k \geq \lceil 1/\eps^2 \rceil$
\begin{align*}
    (C + C\eps^2m)^k
\end{align*}
balls suffice for a suitable constant C.
\end{ex}

\begin{proof}[Solution] 
We can give a tighter bound than given in the proof of Corollary \ref{0.0.4} on the number of ball centres obtained from selecting a set of $k$ points out of $m$ with repetition
(since computing the mean of $k$ is order-invariant with respect to input points). By the ``stars-and-bars''\footnote{\url{https://en.wikipedia.org/wiki/Stars_and_bars_(combinatorics)}} argument, this quantity is given by
\begin{align*}
    \binom{m+k-1}{k -1}
\end{align*}
Note that $\min\{k-1, m\} = k - 1\leq \min\{k, m-1\}$, so looking at row $m+k-1$ of Pascal's triangle
\begin{align*}
    \binom{m+k-1} {k -1} \leq \binom{m+k-1}{k}
\end{align*}
Then, using Exercise \ref{0.0.5}
\begin{align*}
    \binom{m+k-1}{k} \leq \l(\frac {e (m+k-1)}{k}\r)^k = \l(e\frac{k-1}{k} + e\frac1k m\r)^k \leq (e + e\eps^2m)^k
\end{align*}

\end{proof}
